\section{Perl}

\subsection{Organizzazione}

Trattandosi di un sito con una buona quantità di contenuti dinamici, è stato studiato un approccio quanto più modularizzato possibile, in modo da garantire maggior chiarezza e manutenibilità, una sorta di \textit{pattern MVC}, dove le \textit{view} sono rappresentate da templates (\texttt{.tmpl}) raccolti in una directory completamente separata dal codice, modelli e controller sono contenuti in 3 file contenenti inoltre le funzioni principali necessarie al popolamento dinamico del sito, si è quindi resa necessaria la suddivisione di esse in una gerarchia formata da tre moduli:

\begin{itemize}

  \item \texttt{UTILS} classe padre principale, raccoglie le funzioni di uso generale per il funzionamento e la popolazione delle varie pagine, caricamento ed interfaccia dei vari database XML (Model)
  \item \texttt{UTILS::Admin} classe figlio di UTILS, raccoglie le funzioni strettamente necessarie al backend dell'applicazione, funzionalità di login e mantenimento delle sessioni
  \item \texttt{UTILS::UserService} classe figlio di UTILS, raccoglie le funzioni necessarie al compimento delle operazioni strettamente legate all'utente (e.g CRUD delle proprie generalità), prenotazione risorse

\end{itemize}

Alcune funzioni all'interno di questi moduli sono state ``privatizzate'', in quanto funzioni di utilità non direttamente finalizzate all'utilizzo da parte dell'utente (e.g. creazione scheletro tabelle, calcolo e conversione dei giorni della settimana etc.). 
In particolare ognuno di questi moduli fa da appoggio a rispettivi script utilizzati per effettuare le varie operazione per mezzo di dispatch tables, che consentono di risparmiare un gran numero di operazioni ridondanti e di automatizzare il piu possibile le operazioni da eseguire, aumentando inoltre la separazione tra codice e contenuto, avvicinandosi ad un approccio MVC:

\begin{itemize}

  \item \texttt{load.cgi} si appoggia ad \texttt{UTILS} ed è il motore di popolamento principale del sito, ogni pagina accessibile è generata e popolata da questo script, per mezzo di dispatch tables
  \item \texttt{admin.cgi} si appoggia ad \texttt{UTILS::Admin}, controparte backend di \texttt{load.cgi}, ogni pagina della parte amministrativa è generata da questo script
  \item \texttt{process.pl} script necessario alle basilari operazioni di modifica/popolamento risorse/pagine (CRUD)
  \item \texttt{user\_jobs.pl} controparte frontend di \texttt{process.pl}, tutte le operazioni che l'utente può effettuare sono gestite da questo codice

\end{itemize}

Vi sono infine \texttt{login.pl}, \texttt{login.cgi} e \texttt{logout.pl}, piccoli script atti solo all'autenticazione dell'utente, \textit{frontend} e \textit{backend} ed alla chiusura di eventuali sessioni aperte.
\texttt{vbooked.pl} è infine lo script utilizzato per visualizzare le tabelle di prenotazione via AJAX senza il bisogno di effettuare \textit{refresh} della pagina.


\subsection{Sistema di popolamento templates}

Ogni \textit{route} richiama il \textit{dispatcher} da \texttt{UTILS} e passa un \textit{hash} contenente i parametri necessari al popolamento del template richiamato, che inoltre possiede lo stesso nome della \textit{route} appunto.

\scriptsize{\begin{verbatim}
   sub dispatcher {
    my $self = shift;
    my $route = shift;
    my %params = @_;
    my $template = HTML::Template->new(filename => $route.".tmpl", utf8 => 1);
    foreach(keys %params){
        $template->param($_ => $params{$_});
    }	
    my @loop_news = $self->getNews;
    foreach(@loop_news){
        delete $_->{N_ID};
    }
    $template->param(NEWS => \@loop_news);
    print "Content-Type: text/html\n\n", $template->output;
}
\end{verbatim}}
