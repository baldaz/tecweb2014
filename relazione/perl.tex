\section{Perl}

\subsection{Organizzazione}

Trattandosi di un sito con una buona quantità di contenuti dinamici, è stato studiato un approccio quanto più modularizzato possibile, in modo da garantire maggior chiarezza e manutenibilità, una sorta di \textit{pattern MVC}, dove le \textit{view} sono rappresentate da templates (\texttt{.tmpl}) raccolti in una directory completamente separata dal codice, modelli e controller sono contenuti in 3 file contenenti inoltre le funzioni principali necessarie al popolamento dinamico del sito, si è quindi resa necessaria la suddivisione di esse in una gerarchia formata da tre moduli:

\begin{itemize}
  \item \texttt{UTILS}: classe padre principale, raccoglie le funzioni di uso generale per il funzionamento e la popolazione delle varie pagine, caricamento ed interfaccia dei vari database XML (Model)
  \item \texttt{UTILS::Admin}: classe figlio di UTILS, raccoglie le funzioni strettamente necessarie al backend dell'applicazione, funzionalità di login e mantenimento delle sessioni
  \item \texttt{UTILS::UserService}: classe figlio di UTILS, raccoglie le funzioni necessarie al compimento delle operazioni strettamente legate all'utente (e.g CRUD delle proprie generalità), prenotazione risorse
\end{itemize}

La directory \textit{cgi-bin} contiene solo ed esclusivamente script e moduli \textit{perl}, mediante la variabile \$ENV\{HTML\_TEMPLATE\_ROOT\} = "../public\_html/templates"; e' stato possibile mantenere i \textit{templates} in una directory separata (\textit{public\_html/templates/}.

I principali moduli utilizzati sono:

\begin{itemize}
  \item \texttt{CGI}: modulo per la gestione dei parametri input degli script \textit{CGI}
  \item \texttt{CGI::Session}: modulo utilizzato per la gestione delle sessione e lo scambio di dati fra pagine
  \item \texttt{HTML::Template}: modulo per la gestione dei contenuti statici e la popolazione dei contenuti dinamici 
  \item \texttt{XML::LibXML}: modulo per la gestione e l'interfacciamento dei file xml utilizzati per la raccolta delle informazioni e risorse
  \item \texttt{DateTime}: modulo utilizzato nella gestione delle prenotazioni, facilita la manipolazioni delle date e timestamp
\end{itemize}

Alcune funzioni all'interno di questi moduli sono state ``privatizzate'', in quanto funzioni di utilità non direttamente finalizzate all'utilizzo da parte dell'utente (e.g. creazione scheletro tabelle, calcolo e conversione dei giorni della settimana etc.). Esse sono state implementate come \texttt{subroutine} anonime assegnate a variabili, in modo che possano essere richiamate solo all' interno dei moduli in cui sono dichiarate; nella fattispecie, e' stata utilizzata una funzione protetta per la creazione delle tabelle di prenotazione dinamiche. 
In particolare ognuno di questi moduli fa da appoggio a rispettivi script utilizzati per effettuare le varie operazione per mezzo di dispatch tables, che consentono di risparmiare un gran numero di operazioni ridondanti e di automatizzare il piu possibile le operazioni da eseguire, aumentando inoltre la separazione tra codice e contenuto, avvicinandosi ad un approccio MVC:

\begin{itemize}

  \item \texttt{load.cgi}: si appoggia ad \texttt{UTILS} ed è il motore di popolamento principale del sito, ogni pagina accessibile è generata e popolata da questo script, per mezzo di dispatch tables
  \item \texttt{admin.cgi}: si appoggia ad \texttt{UTILS::Admin}, controparte backend di \texttt{load.cgi}, ogni pagina della parte amministrativa è generata da questo script
  \item \texttt{process.pl}: script necessario alle basilari operazioni di modifica/popolamento risorse/pagine (CRUD)
  \item \texttt{user\_jobs.pl}: controparte frontend di \texttt{process.pl}, tutte le operazioni che l'utente può effettuare sono gestite da questo codice
\end{itemize}

Vi sono infine \texttt{login.pl}, \texttt{login.cgi} e \texttt{logout.pl}, piccoli script atti solo all'autenticazione dell'utente, \textit{frontend} e \textit{backend} ed alla chiusura di eventuali sessioni aperte.
\texttt{vbooked.pl} è infine lo script utilizzato per visualizzare le tabelle di prenotazione via AJAX senza il bisogno di effettuare \textit{refresh} della pagina.


\subsection{Sistema di popolamento templates}

Ogni \textit{route} richiama il \textit{dispatcher} da \texttt{UTILS} e passa un \textit{hash} contenente i parametri necessari al popolamento del template richiamato, che inoltre possiede lo stesso nome della \textit{route} appunto.\newline
Da \texttt{load.cgi} attraverso l'oggetto \texttt{\$utils} e la dispatch table viene automaticamente richiamato e popolato il template corretto: \newline \newline

\small{\textit{Dispatch table} all'interno di \texttt{load.cgi}:}


\scriptsize{
\begin{verbatim}

    my %routes = (
      'home'          => \&index,
      'impianti'      => \&impianti,
      'contatti'      => \&contatti,
      'corsi'         => \&corsi,
      'prenotazioni'  => \&prenotazioni,
      'registrazione' => \&registrazione,
      'personale'     => \&personale,
      'prenota'       => \&prenota,
      'edit_personal' => \&edit_personal
    );

    if( grep { $page eq $_} keys %routes){
      $routes{$page}->();
    }

\end{verbatim}
}

\small{Funzione \texttt{corsi} associata alla route corsi:}

\scriptsize{
\begin{verbatim}
   sub corsi {
     my @loop_prices = $utils->list_prices;
     my @loop_scheduling = $utils->list_scheduling;
     my %params = (
        title => 'Centro sportivo - Corsi',
        page      => 'corsi',
        path      => 'Corsi',
        courses_price => \@loop_prices,
        courses_scheduling => \@loop_scheduling,
        LOGIN     => $sess_params{is_logged},
        USER      => $sess_params{profile},
        attempt   => $sess_params{attempt}
        );
     $utils->dispatcher('corsi', %params);
   }
\end{verbatim}
}
\small{Funzione \texttt{dispatcher} all'interno di \texttt{UTILS.pm}, avendo per convenzione \texttt{\$route} il nome del template a cui la route è associata, esso viene richiamato e popolato con i parametri contenuti in \texttt{\%params} settati nella funzione \texttt{corsi} in \texttt{load.cgi}:}
 
\scriptsize{
\begin{verbatim}
   sub dispatcher {
     my $self = shift;
     my $route = shift;
     my %params = @_;
     my $template = HTML::Template->new(filename => $route.".tmpl", utf8 => 1);
     foreach(keys %params){
        $template->param($_ => $params{$_});
     }	
     my @loop_news = $self->getNews;
     foreach(@loop_news){
        delete $_->{N_ID};
     }
     $template->param(NEWS => @loop_news);
     print "Content-Type: text/html\n\n", $template->output;
   }
\end{verbatim}
}
Ogni input e' stato validato oltre che lato client via javascript, anche lato server, utilizzando funzioni di validazione appositamente create, in modo da garantire procedure di login, registrazione, prenotazione e modifica prive di spiacevoli errori.
