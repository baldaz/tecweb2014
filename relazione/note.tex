\section{Note e difficolta' riscontrate}
\subsection{Difficolta'}
Il problema piu' grosso che abbiamo avuto e' stato il fatto di dover completare il progetto in due membri anziche' tre.
Alcune feature come ad esempio una traduzione in inglese sono state tralasciate per via del elevato carico di lavoro.\newline
Un iniziale scelta dello schema di colori nel layout ha creato problemi nella parte di accessibilita' e siamo stati costretti a cambiarlo.

\subsection{Note}
La parte amministrativa e' rudimentale e un po' scarna sempre per la mancanza di tempo, abbiamo pensato di fare solo una validazione tramite Javascript in questa sezione poiche' si presume che l'amministratore del sito sia informato a dovere e abbia un idea giusta del contenuto che andra' a mettere nel proprio sito (ad esempio nei corsi evitare di sovrascrivere gli orari in qualche giorno e' sua responsabilita').\newline
Abbiamo deciso di separare la parte amministrativa dal sito per scelte di sicurezza (\textit{information hiding}).
Nel nostro sito l'utilizzo di \texttt{javascript} e' limitato alla validazione dei dati nei vari form e al sistema di visualizzazione delle tabelle di prenotazione senza bisogno di refresh (\texttt{AJAX}).\newline
Sono stati effettuati i test nel caso venga disabilitato \texttt{javascript} nel sito, come fallback ci sono controlli lato server via \texttt{Perl} e un button per il refresh della pagina per la visualizzazione delle tabelle di prenotazione sempre via \texttt{Perl}.\newline 
Come "database" per le prenotazioni per i profili e tutto cio' che serviva immagazzinare abbiamo usato dei file \texttt{XML}, in tali files non abbiamo inserito i namespace tuttavia sono provvisti di \textit{schema} e sono stati validati online. 
