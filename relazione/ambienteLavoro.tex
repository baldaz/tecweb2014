\section{Ambiente di Lavoro}

Per la cooperazione nello sviluppo del progetto si e' scelto di usare \textbf{github}. 
Il progetto e' stato quindi messo in una repository \textbf{git} in modo da poter essere condiviso da entrambi i componenti del gruppo.
Tuttavia per la suddivisione dei compiti e per le parti realizzate in cooperativa abbiamo lavorato piu' giorni assieme nel laboratorio universitario in Paolotti.
Per quasi la totalita' del progetto si e' lavorato su ambiente \textbf{linux} con qualche eccezione per i vari test sia per l'accessibilita' sia per layout su diverso OS.
Alcuni dei programmi utilizzati per lo sviluppo sono:
\begin{itemize}
	\item	per la scrittura del codice e' stato utilizzato \textbf{emacs} e \textbf{sublime text}
	\item come server locale per la prova e lo sviluppo da casa \textbf{apache2}
	\item per la gestione e modifica delle immagini \textbf{gimp}
	\item per la relazione \textbf{LateX}
\end{itemize}
